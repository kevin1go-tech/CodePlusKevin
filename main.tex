\documentclass[14pt]{article}
\usepackage[utf8]{inputenc}
\usepackage{graphicx}
\usepackage{lipsum}
%package list
\usepackage[top=3cm, bottom=3cm, outer=3cm, inner=3cm]{geometry}
\usepackage{multicol}
\usepackage{url}
%\usepackage{cite}
\usepackage{hyperref}
\usepackage{array}
%\usepackage{multicol}
\newcolumntype{x}[1]{>{\centering\arraybackslash\hspace{0pt}}p{#1}}
\usepackage{natbib}
\usepackage{pdfpages}
\usepackage{multirow}
\usepackage[normalem]{ulem}
\useunder{\uline}{\ul}{}
\usepackage{svg}
\usepackage{xcolor}
\usepackage{listings}
\lstdefinestyle{ascii-tree}{
    literate={├}{|}1 {─}{--}1 {└}{+}1 
  }
\lstset{basicstyle=\ttfamily,
  showstringspaces=false,
  commentstyle=\color{red},
  keywordstyle=\color{blue}
}
%\usepackage{booktabs}
\usepackage{caption}
\usepackage{subcaption}
\usepackage{float}
\usepackage{array}

\newcolumntype{M}[1]{>{\centering\arraybackslash}m{#1}}
\newcolumntype{N}{@{}m{0pt}@{}}


%%%%%%%%%%%%%%%%%%%%%%%%%%%%%%%%%%%%%%%%%%%%%%%%%%%%%%%%%%%%%%%%%%%%%%%%%%%%
%%%%%%%%%%%%%%%%%%%%%%%%%%%%%%%%%%%%%%%%%%%%%%%%%%%%%%%%%%%%%%%%%%%%%%%%%%%%
\newcommand{\itemEmail}{Kgonzalesf@ulasalle.edu.pe}
\newcommand{\itemStudent}{Kevin Gonzales Fernandez}
\newcommand{\itemCourse}{Ingeniería Web}
\newcommand{\itemCourseCode}{3.8.6.16}
\newcommand{\itemSemester}{VIII}
\newcommand{\itemUniversity}{Universidad La Salle}
\newcommand{\itemFaculty}{Facultad de Ingenierías}
\newcommand{\itemDepartment}{Departamento de Ingeniería y Matemáticas}
\newcommand{\itemSchool}{Carrera Profesional de Ingeniería de Software}
\newcommand{\itemAcademic}{2023 - B}
\newcommand{\itemInput}{28 Agosto 2023}
\newcommand{\itemOutput}{02 Setiembre 2023}
\newcommand{\itemPracticeNumber}{02}
\newcommand{\itemTheme}{JavaScript}
%%%%%%%%%%%%%%%%%%%%%%%%%%%%%%%%%%%%%%%%%%%%%%%%%%%%%%%%%%%%%%%%%%%%%%%%%%%%
%%%%%%%%%%%%%%%%%%%%%%%%%%%%%%%%%%%%%%%%%%%%%%%%%%%%%%%%%%%%%%%%%%%%%%%%%%%%

\usepackage[english,spanish]{babel}
\usepackage[utf8]{inputenc}
\AtBeginDocument{\selectlanguage{spanish}}
\renewcommand{\figurename}{Figura}
\renewcommand{\refname}{Referencias}
\renewcommand{\tablename}{Tabla} %esto no funciona cuando se usa babel
\AtBeginDocument{%
	\renewcommand\tablename{Tabla}
}

\usepackage{fancyhdr}
\pagestyle{fancy}
\fancyhf{}
\setlength{\headheight}{30pt}
\renewcommand{\headrulewidth}{1pt}
\renewcommand{\footrulewidth}{1pt}
\fancyhead[L]{\raisebox{-0.2\height}{\includegraphics[width=3cm]{logo_salle.png}}}
\fancyhead[C]{\fontsize{7}{7}\selectfont	\itemUniversity \\ \itemFaculty \\ \itemDepartment \\ \itemSchool \\ \textbf{\itemCourse}}
%\fancyhead[R]{\raisebox{-0.2\height}{\includegraphics[width=1.2cm]{img/logo_salle}}}
\fancyfoot[L]{M.Sc. Ing. Richart Smith Escobedo Quispe}
\fancyfoot[C]{Página \thepage}
\fancyfoot[R]{\itemCourse}

% para el codigo fuente
\usepackage{listings}
\usepackage{color, colortbl}
\definecolor{dkgreen}{rgb}{0,0.6,0}
\definecolor{gray}{rgb}{0.5,0.5,0.5}
\definecolor{mauve}{rgb}{0.58,0,0.82}
\definecolor{codebackground}{rgb}{0.95, 0.95, 0.92}
\definecolor{tablebackground}{rgb}{0.0, 0.45, 0.63}

\lstset{frame=tb,
	language=bash,
	aboveskip=3mm,
	belowskip=3mm,
	showstringspaces=false,
	columns=flexible,
	basicstyle={\small\ttfamily},
	numbers=none,
	numberstyle=\tiny\color{gray},
	keywordstyle=\color{blue},
	commentstyle=\color{dkgreen},
	stringstyle=\color{mauve},
	breaklines=true,
	breakatwhitespace=true,
	tabsize=3,
	backgroundcolor= \color{codebackground},
}

\begin{document}
	
	\vspace*{10px}
	
	\begin{center}	
		\fontsize{17}{17} \textbf{ Laboratorio \itemPracticeNumber}
	\end{center}
	\centerline{\textbf{\Large Tema: \itemTheme}}
	%\vspace*{0.5cm}	

	\begin{flushright}
		\begin{tabular}{|M{2.5cm}|N|}
			\hline 
			\rowcolor{tablebackground}
			\color{white} \textbf{Nota}  \\
			\hline 
			     \\[30pt]
			\hline 			
		\end{tabular}
	\end{flushright}	

	\begin{table}[H]
		\begin{tabular}{|x{4.7cm}|x{4.8cm}|x{4.8cm}|}
			\hline 
			\rowcolor{tablebackground}
			\color{white} \textbf{Estudiante} & \color{white}\textbf{Escuela}  & \color{white}\textbf{Asignatura}   \\
			\hline 
			{\itemStudent \par \itemEmail} & \itemSchool & {\itemCourse \par Semestre: \itemSemester \par Código: \itemCourseCode}     \\
			\hline 			
		\end{tabular}
	\end{table}		
	
	\begin{table}[H]
		\begin{tabular}{|x{4.7cm}|x{4.8cm}|x{4.8cm}|}
			\hline 
			\rowcolor{tablebackground}
			\color{white}\textbf{Laboratorio} & \color{white}\textbf{Tema}  & \color{white}\textbf{Duración}   \\
			\hline 
			\itemPracticeNumber & \itemTheme & 06 horas   \\
			\hline 
		\end{tabular}
	\end{table}
	
	\begin{table}[H]
		\begin{tabular}{|x{4.7cm}|x{4.8cm}|x{4.8cm}|}
			\hline 
			\rowcolor{tablebackground}
			\color{white}\textbf{Semestre académico} & \color{white}\textbf{Fecha de inicio}  & \color{white}\textbf{Fecha de entrega}   \\
			\hline 
			\itemAcademic & \itemInput &  \itemOutput  \\
			\hline 
		\end{tabular}
	\end{table}

    \section{Tarea}
    \subsection{Ejercicio 01}
    Cree una versión de el juego ’el ahorcado’ que grafique con canvas paso a paso desde el evento onclick() de un botón.
    \subsubsection{Codigo HTML}
    En el código Html se mostrara la estructura de la página web, ubicando y cargando los códigos Css junto al javascript, para luego organizar los lugares y funciones donde estará ubicado cada cosa del juego del ahorcado.

    \begin{lstlisting}[language=html,caption={Code HTML}][H]
		<!DOCTYPE html>
<html lang="es">
<head>
    <meta charset="UTF-8">
    <meta name="viewport" content="width=device-width, initial-scale=1.0">
    <title>El Juego Interactivo del Ahorcado</title>
    <link rel="stylesheet" href="style.css">
</head>
<body>
    <div class="header">
        <h1>El juego interactivo del ahorcado</h1>
    </div>
    <div class="container">
        <canvas id="canvas" width="400" height="400"></canvas>
        <button id="playButton">Presionar</button>
    </div>
    <div id="winModal" class="modal">
        <div class="modal-content">
            <span class="close">&times;</span>
            <p>Felicidades, ¡has ganado!</p>
            <button id="restartButton">Comenzar de Nuevo</button>
        </div>
    </div>
    <script src="script.js"></script>
</body>
</html>

	\end{lstlisting}

    \subsubsection{Codigo CSS}
    En el código Css se nos muestra los colores y diseños que se usara en la pagina web del juego del ahorcado, dándole colores a los botones y al fondo de esta misma. 

    \begin{lstlisting}[language=HTML,caption={Code CSS}][H]
    body {
    margin: 0;
    padding: 0;
    background-color: lightgreen;
}

.header {
    background-color: darkgreen;
    text-align: center;
    padding: 20px 0;
}

h1 {
    color: #fbff00;
    font-size: 20px;
}

.container {
    text-align: center;
}

#canvas {
    background-color: lightgreen;
    border: 2px solid darkgreen;
    margin-top: 20px;
}

#playButton {
    position: absolute; /* Añadido para posicionar el botón */
    bottom: 10px; /* Ajuste para posicionar debajo de la imagen */
    left: 50%; /* Añadido para centrar horizontalmente */
    transform: translateX(-50%); /* Añadido para centrar horizontalmente */
    padding: 10px 20px;
    font-size: 16px;
    background-color: red;
    color: white;
    border: none;
    cursor: pointer;
}

.modal {
    display: none;
    position: fixed;
    z-index: 1;
    left: 0;
    top: 0;
    width: 100%;
    height: 100%;
    background-color: rgba(0,0,0,0.5);
}

.modal-content {
    background-color: white;
    margin: 25% auto;
    padding: 20px;
    border: 1px solid #888;
    width: 50%;
}

.close {
    color: #aaa;
    float: right;
    font-size: 28px;
    font-weight: bold;
    cursor: pointer;
}

.close:hover,
.close:focus {
    color: black;
    text-decoration: none;
    cursor: pointer;
}
#restartButton {
    padding: 10px 20px;
    font-size: 16px;
    background-color: orange;
    color: #ffffff;
    border: none;
    cursor: pointer;
}

    \end{lstlisting}
    \subsubsection{Codigo Javascript}
    En el código javascript se mostrara todas las mecánicas y funciones que darán vida al juego del ahorcado, como la de presionar los botones, el canvas y la ventana extra.
    \begin{lstlisting} [language=html,caption={Code HTML}][H]
    document.addEventListener("DOMContentLoaded", function() {
    const canvas = document.getElementById('canvas');
    const ctx = canvas.getContext('2d');
    const playButton = document.getElementById('playButton');
    const winModal = document.getElementById('winModal');
    const restartButton = document.getElementById('restartButton');
    const closeButton = document.querySelector('.close');

    const drawHanger = () => {
        ctx.beginPath();
        ctx.moveTo(50, 350);
        ctx.lineTo(150, 350);
        ctx.lineTo(100, 300);
        ctx.lineTo(50, 350);
        ctx.lineTo(50, 50);
        ctx.lineTo(200, 50);
        ctx.lineTo(200, 100);
        ctx.stroke();
    };

    const drawMan = (step) => {
        switch (step) {
            case 1:
                ctx.beginPath();
                ctx.arc(200, 150, 50, 0, Math.PI * 2);
                ctx.stroke();
                break;
            case 2:
                ctx.beginPath();
                ctx.moveTo(200, 200);
                ctx.lineTo(200, 300);
                ctx.stroke();
                break;
            case 3:
                ctx.beginPath();
                ctx.moveTo(200, 220);
                ctx.lineTo(150, 250);
                ctx.stroke();
                break;
            case 4:
                ctx.beginPath();
                ctx.moveTo(200, 220);
                ctx.lineTo(250, 250);
                ctx.stroke();
                break;
            case 5:
                ctx.beginPath();
                ctx.moveTo(200, 300);
                ctx.lineTo(150, 350);
                ctx.stroke();
                break;
            case 6:
                ctx.beginPath();
                ctx.moveTo(200, 300);
                ctx.lineTo(250, 350);
                ctx.stroke();
                break;
            default:
                break;
        }
    };

    const drawGameOver = () => {
        winModal.style.display = 'block';
    };

    const clearCanvas = () => {
        ctx.clearRect(0, 0, canvas.width, canvas.height);
        drawHanger();
    };

    let gameStarted = false;
    let gameOver = false;
    let step = 0;

    playButton.addEventListener('click', () => {
        if (!gameOver) {
            if (!gameStarted) {
                clearCanvas();
                gameStarted = true;
            } else {
                step++;
                drawMan(step);
                if (step >= 6) {
                    drawGameOver();
                    gameOver = true;
                }
            }
        }
    });

    restartButton.addEventListener('click', () => {
        winModal.style.display = 'none';
        gameStarted = false;
        gameOver = false;
        step = 0;
        clearCanvas();
    });

    closeButton.addEventListener('click', () => {
        winModal.style.display = 'none';
    });
});
    \end{lstlisting}
    \vspace{100pt}
    \subsubsection{Ejecución de la página web}
    Procederemos a mostrar como funciona la pagina web del ahorcado, para jugar tendrás que presionar el botón rojo, el cual hará que se vaya construyendo de poco en poco el ahorcado, para cuando lo completes te saldrá una ventana de felicidades.


    \begin{figure}[H]
    \centering
    \includegraphics[scale=0.46]{Ahorcado.png}
    \caption{Ejecución del juego Ahorcado} %
    \label{fig:imagen} 
    \end{figure}

    \subsection{Ejercicio 02}
    Cree una calculadora básica como la de los sistemas operativos, que pueda utilizar la función eval() y que guarde todos las operaciones en una pila. Mostrar la pila al píe de la página web. 

    \subsubsection{Codigo HTML}
    En esta parte del código mostraremos la pagina web en el navegador, para ello el código recoge los códigos Css y javascript para mostrar los colores y funciones de la calculadora.
    
    \begin{lstlisting} [language=html,caption={Code HTML}][H]
    <!DOCTYPE html>
<html lang="es">
<head>
    <meta charset="UTF-8">
    <meta name="viewport" content="width=device-width, initial-scale=1.0">
    <title>Calculadora con Pila</title>
    <link rel="stylesheet" href="style.css">
</head>
<body>
    <div class="barra"></div>
    <div class="calculadora">
        <input type="text" id="display" readonly>
        <div class="botones">
            <button onclick="agregarCaracter('(')">(</button>
            <button onclick="agregarCaracter(')')">)</button>
            <button onclick="limpiarDisplay()">C</button>
            <button onclick="borrarCaracter()">Borrar</button>
            <button onclick="agregarCaracter('pi')">π</button>
            <button onclick="agregarCaracter('Math.sqrt(')">√</button>
            <button onclick="agregarCaracter('%')">%</button>
            <button onclick="agregarCaracter('n')">n</button>
            <button onclick="agregarCaracter('1')">1</button>
            <button onclick="agregarCaracter('2')">2</button>
            <button onclick="agregarCaracter('3')">3</button>
            <button onclick="agregarCaracter('+')">+</button>
            <button onclick="agregarCaracter('4')">4</button>
            <button onclick="agregarCaracter('5')">5</button>
            <button onclick="agregarCaracter('6')">6</button>
            <button onclick="agregarCaracter('-')">-</button>
            <button onclick="agregarCaracter('7')">7</button>
            <button onclick="agregarCaracter('8')">8</button>
            <button onclick="agregarCaracter('9')">9</button>
            <button onclick="agregarCaracter('*')">*</button>
            <button onclick="agregarCaracter('0')">0</button>
            <button onclick="agregarCaracter('.')">.</button>
            <button onclick="calcular()">=</button>
            <button onclick="agregarCaracter('/')">/</button>
            <button onclick="agregarCaracter('**2')">X²</button>
            <button onclick="agregarCaracter('**')">x</button>
            <button onclick="agregarCaracter('Math.E')">e</button>
        </div>
    </div>
    <div class="pilas">
        <div class="lista-pilas" id="lista-pilas">
            <!-- Aquí se mostrarán las pilas -->
        </div>
        <button onclick="borrarPila()">Borrar Pila</button>
        <button onclick="editarPila()">Editar Pila</button>
    </div>
    <div id="modal" class="modal">
        <div class="modal-content">
            <span class="close" onclick="cerrarModal()">&times;</span>
            <p id="modal-title"></p>
            <textarea id="modal-input"></textarea>
            <button onclick="confirmarModal()">Confirmar</button>
        </div>
    </div>
    <script src="script.js"></script>
</body>
</html>

    \end{lstlisting}

    \subssubection{Codigo CSS}
    En esta parte del código mostraremos los diseños, colores y botones que se muestra en la calculadora, también regulando su tamaño y ancho.
    \begin{lstlisting} [language=html,caption={Code CSS}][H]
    body {
    margin: 0;
    padding: 0;
    background-color: #eaf2e3; /* Verde claro */
}

.barra {
    width: 100%;
    height: 50px;
    background-color: #2e7d32; /* Verde oscuro */
}

.calculadora {
    margin: 20px auto;
    width: 300px;
    background-color: #fff;
    padding: 20px;
    border-radius: 10px;
    box-shadow: 0px 0px 10px rgba(0, 0, 0, 0.1);
}

#display {
    width: calc(100% - 40px);
    height: 40px;
    margin-bottom: 10px;
    font-size: 20px;
    padding: 5px;
    border: 1px solid #ccc;
    border-radius: 5px;
}

.botones {
    display: grid;
    grid-template-columns: repeat(4, 1fr);
    gap: 5px;
}

button {
    width: calc(100% - 10px);
    height: 40px;
    font-size: 18px;
    border: none;
    border-radius: 5px;
    background-color: #2e7d32;
    color: #fff;
    cursor: pointer;
}

button:hover {
    background-color: #5cb85c;
}

.pilas {
    margin-top: 20px;
    padding: 20px;
    background-color: #fff;
    border-radius: 10px;
    box-shadow: 0px 0px 10px rgba(0, 0, 0, 0.1);
}

.lista-pilas {
    max-height: 200px;
    overflow-y: auto;
    border: 1px solid #ccc;
    border-radius: 5px;
    padding: 10px;
    margin-bottom: 10px;
}

.lista-pilas p {
    margin: 5px 0;
}

.modal {
    display: none;
    position: fixed;
    z-index: 1;
    left: 0;
    top: 0;
    width: 100%;
    height: 100%;
    background-color: rgba(0, 0, 0, 0.5);
}

.modal-content {
    background-color: #fefefe;
    margin: 20% auto;
    padding: 20px;
    border-radius: 10px;
    box-shadow: 0px 0px 10px rgba(0, 0, 0, 0.5);
}

.close {
    color: #aaa;
    float: right;
    font-size: 28px;
    font-weight: bold;
}

.close:hover,
.close:focus {
    color: black;
    text-decoration: none;
    cursor: pointer;
}

#modal-input {
    width: calc(100% - 20px);
    height: 100px;
    margin-top: 10px;
    padding: 10px;
    border: 1px solid #ccc;
    border-radius: 5px;
}

    \end{lstlisting}

    \subsubsection{Codigo Javascript}
    En esta parte del código se mostrará todas las mecánicas y funciones que se pusieron, como la lógica de la calculadora como los botones y las pilas.
    \begin{lstlisting} [language=html,caption={Code Javasript}][H]
    let display = document.getElementById('display');
let listaPilas = document.getElementById('lista-pilas');
let operaciones = [];

function agregarCaracter(caracter) {
    display.value += caracter;
}

function borrarCaracter() {
    display.value = display.value.slice(0, -1);
}

function limpiarDisplay() {
    display.value = '';
}

function calcular() {
    try {
        let resultado = eval(display.value);
        operaciones.push({ operacion: display.value, resultado: resultado });
        mostrarPilas();
        display.value = resultado;
    } catch (error) {
        display.value = 'Error';
    }
}

function mostrarPilas() {
    listaPilas.innerHTML = '';
    for (let i = 0; i < operaciones.length; i++) {
        let p = document.createElement('p');
        p.textContent = operaciones[i].operacion + ' = ' + operaciones[i].resultado;
        listaPilas.appendChild(p);
    }
}

function borrarPila() {
    let indicePila = prompt('Ingrese el índice de la pila a borrar (comenzando desde 1):');
    if (indicePila && !isNaN(indicePila) && indicePila > 0 && indicePila <= operaciones.length) {
        operaciones.splice(indicePila - 1, operaciones.length - indicePila + 1);
        mostrarPilas();
    } else {
        alert('Índice de pila inválido');
    }
}

function editarPila() {
    let indicePila = prompt('Ingrese el índice de la pila a editar (comenzando desde 1):');
    if (indicePila && !isNaN(indicePila) && indicePila > 0 && indicePila <= operaciones.length) {
        mostrarModal('Editar Pila', 'Ingrese la nueva operación:', (nuevaOperacion) => {
            operaciones.splice(indicePila - 1, operaciones.length - indicePila + 1);
            calcularNuevaOperacion(nuevaOperacion);
        });
    } else {
        alert('Índice de pila inválido');
    }
}

function calcularNuevaOperacion(nuevaOperacion) {
    try {
        let resultado = eval(nuevaOperacion);
        operaciones.push({ operacion: nuevaOperacion, resultado: resultado });
        mostrarPilas();
        display.value = resultado;
    } catch (error) {
        display.value = 'Error';
    }
}

function mostrarModal(titulo, placeholder, callback) {
    let modal = document.getElementById('modal');
    let modalTitle = document.getElementById('modal-title');
    let modalInput = document.getElementById('modal-input');

    modal.style.display = 'block';
    modalTitle.textContent = titulo;
    modalInput.placeholder = placeholder;

    modalInput.value = '';
    modalInput.focus();

    let confirmarModal = document.querySelector('.modal-content button');
    confirmarModal.onclick = function () {
        modal.style.display = 'none';
        callback(modalInput.value);
    };
}

function cerrarModal() {
    let modal = document.getElementById('modal');
    modal.style.display = 'none';
}
    \end{lstlisting}

    \subsubsection{Ejecución de la página web}
    Procederemos a mostrar el funcionamiento de la pagina web, al iniciar, se nos mostrara una calculadora donde podras realizar cálculos matemáticos, estos se iran guardando en pilas, las cuales puedes editar y borra según tu agrado
    
    \begin{figure}[H]
    \centering
    \includegraphics[scale=0.46]{Calculadora.png}
    \caption{Ejecución de la calculadora} %
    \label{fig:imagen} 
    \end{figure}

    \subsection{Ejercicio 03}
    Cree un teclado random para banca por internet

    \subsubsection{Codigo HTML}
    En esta parte del código mostraremos la pagina web, vinculando los códigos css de diseño y javascript de funcionamiento, para que la hora de mostrarse la pagina web aparezca todo ordenado y con las condiciones que le pusiste.
    \begin{lstlisting}[language=html,caption={Code HTML}][H]
    <!DOCTYPE html>
<html lang="es">
<head>
    <meta charset="UTF-8">
    <meta name="viewport" content="width=device-width, initial-scale=1.0">
    <title>Página Web</title>
    <link rel="stylesheet" href="style.css">
</head>
<body>
    <div class="header">
        <img src="Imagenes/logo.png" alt="Logo" class="logo-left">
        <img src="Imagenes/logoA.png" alt="LogoA" class="logo-right">
    </div>
    <div class="container">
        <div class="content">
            <div class="form">
                <p class="security-text">Se Encuentra en una Zona Segura</p>
                <label for="cardType">Seleccion:</label>
                <select name="cardType" id="cardType">
                    <option value="Multired global debito">Multired global debito</option>
                    <option value="Electronico debito">Electronico debito</option>
                    <option value="Monedero debito">Monedero debito</option>
                    <option value="debito">debito</option>
                    <option value="otros">otros</option>
                </select>
                <label for="cardNumber">Numero de targeta:</label>
                <input type="text" id="cardNumber" maxlength="16">
                <label for="documentType">Tipo de documento:</label>
                <select name="documentType" id="documentType">
                    <option value="DNI">DNI</option>
                    <option value="Pasaporte">Pasaporte</option>
                </select>
                <input type="text" id="documentNumber" maxlength="12" placeholder="Número de documento">
                <label for="internetKey">Generar tu clave por internet (solo hasta 6 dígitos):</label>
                <input type="password" id="internetKey" maxlength="6" placeholder="Ingresa tu clave de internet" readonly>
                <div class="virtual-keyboard">
                    <!-- Teclado virtual -->
                </div>
                <label for="captcha">Ingresar captcha:</label>
                <div class="captcha-row">
                    <div class="captcha">
                        <!-- Captcha -->
                    </div>
                    <button id="captchaRefreshButton">Refrescar</button>
                </div>
                <input type="text" id="captchaInput" placeholder="Ingresa el captcha">
                <button id="submitButton">Ingresar</button>
            </div>
        </div>
    </div>
    <div id="successModal" class="modal">
        <div class="modal-content">
            <span class="close">&times;</span>
            <p>Tus datos fueron registrados. Gracias por compartir tu información.</p>
            <button id="reloadButton">Volver a cargar datos</button>
        </div>
    </div>
    <script src="script.js"></script>
</body>
</html>
    \end{lstlisting}

    \subsubsection{Codigo CSS}
    En esta parte del código se muestra los diseños, colores y matices que se usaron para darle color y forma a la pagina web, controlando estrictamente su tamaño y posicion.
    \begin{lstlisting}[language=html,caption={Code CSS}][H]
    body {
    margin: 0;
    padding: 0;
    background-color: lightblue;
    font-family: Arial, sans-serif;
    text-transform: capitalize;
}

.header {
    background-color: green;
    padding: 10px;
    display: flex;
    justify-content: space-between;
    align-items: center;
}

.logo-left {
    width: 100px;
    height: auto;
}

.logo-right {
    width: 100px;
    height: auto;
}

.container {
    max-width: 600px;
    margin: 20px auto;
    background-color: white;
    padding: 20px;
    border-radius: 10px;
    box-shadow: 0 0 10px rgba(0, 0, 0, 0.1);
}

.content {
    display: flex;
    justify-content: center;
    align-items: center;
}

.form {
    margin-left: 20px;
}

.security-text {
    color: red;
}

label {
    display: block;
    margin-bottom: 5px;
}

input[type="text"],
select,
input[type="password"] {
    width: calc(100% - 20px);
    padding: 8px;
    margin-bottom: 10px;
    border: 1px solid #ccc;
    border-radius: 5px;
}

input[type="text"]:focus,
select:focus,
input[type="password"]:focus {
    outline: none;
    border-color: blue;
}

.virtual-keyboard {
    display: grid;
    grid-template-columns: repeat(3, 1fr);
    gap: 5px;
    margin-bottom: 10px;
}

.virtual-keyboard button {
    width: 100%;
    height: 40px;
    border: 1px solid #ccc;
    border-radius: 5px;
    background-color: #fff;
    cursor: pointer;
}

.captcha-row {
    display: flex;
    align-items: center;
    margin-bottom: 10px;
}

.captcha {
    flex: 1;
    display: flex;
    align-items: center;
}

.captcha img {
    margin-right: 10px;
}

#captchaRefreshButton {
    margin-left: 10px;
}

#submitButton {
    background-color: red;
    color: white;
    border: none;
    padding: 10px 20px;
    border-radius: 5px;
    cursor: pointer;
}

.modal {
    display: none;
    position: fixed;
    z-index: 1;
    left: 50%;
    top: 50%;
    transform: translate(-50%, -50%);
    width: 50%;
    max-width: 400px;
    background-color: rgba(0, 0, 0, 0.5);
    text-align: center;
    padding: 20px;
}

.modal-content {
    background-color: white;
    padding: 20px;
    border-radius: 10px;
}

.close {
    color: #aaa;
    float: right;
    font-size: 28px;
    font-weight: bold;
    cursor: pointer;
}

.close:hover,
.close:focus {
    color: black;
    text-decoration: none;
    cursor: pointer;
}

#reloadButton {
    background-color: red;
    color: white;
    border: none;
    padding: 10px 20px;
    border-radius: 5px;
    cursor: pointer;
}

#reloadButton:hover {
    background-color: darkred;
}

    \end{lstlisting}

    \subsubsection{Codigo Javascript}
    En esta parte del código se nos mostrara todas las funcionalidades y mecánicas que tiene nuestra pagina web, como los botones, selección y escritura, ademas de guardar los datos.
    \begin{lstlisting}[language=html,caption={Code Javasript}][H]
    document.addEventListener("DOMContentLoaded", function() {
    const captchaContainer = document.querySelector('.captcha');
    const captchaInput = document.getElementById('captchaInput');
    const captchaRefreshButton = document.getElementById('captchaRefreshButton');
    const submitButton = document.getElementById('submitButton');
    const successModal = document.getElementById('successModal');
    const reloadButton = document.getElementById('reloadButton');
    const cardTypeSelect = document.getElementById('cardType');
    const cardNumberInput = document.getElementById('cardNumber');
    const documentTypeSelect = document.getElementById('documentType');
    const documentNumberInput = document.getElementById('documentNumber');
    const internetKeyInput = document.getElementById('internetKey');
    const virtualKeyboard = document.querySelector('.virtual-keyboard');

    const generateCaptcha = () => {
        const characters = 'ABCDEFGHIJKLMNOPQRSTUVWXYZabcdefghijklmnopqrstuvwxyz0123456789';
        let captcha = '';
        for (let i = 0; i < 6; i++) {
            captcha += characters.charAt(Math.floor(Math.random() * characters.length));
        }
        return captcha;
    };

    let currentCaptcha = generateCaptcha();
    captchaContainer.innerHTML = `<img src="captcha.php?captcha=${currentCaptcha}" alt="Captcha">${currentCaptcha}`;

    captchaRefreshButton.addEventListener('click', () => {
        currentCaptcha = generateCaptcha();
        captchaContainer.innerHTML = `<img src="captcha.php?captcha=${currentCaptcha}" alt="Captcha">${currentCaptcha}`;
    });

    const handleSubmit = () => {
        successModal.style.display = 'block';
    };

    submitButton.addEventListener('click', handleSubmit);

    reloadButton.addEventListener('click', () => {
        successModal.style.display = 'none';
        cardNumberInput.value = '';
        documentNumberInput.value = '';
        internetKeyInput.value = '';
        generateCaptchaImage();
    });

    documentTypeSelect.addEventListener('change', () => {
        const selectedOption = documentTypeSelect.value;
        if (selectedOption === 'DNI') {
            documentNumberInput.placeholder = 'Número de DNI (8 dígitos)';
            documentNumberInput.maxLength = 8;
        } else if (selectedOption === 'Pasaporte') {
            documentNumberInput.placeholder = 'Número de Pasaporte (12 dígitos)';
            documentNumberInput.maxLength = 12;
        }
    });

    const keyboardButtons = Array.from(virtualKeyboard.querySelectorAll('button'));

    keyboardButtons.forEach(button => {
        button.addEventListener('click', () => {
            if (button.textContent === 'Limpiar') {
                internetKeyInput.value = '';
            } else {
                internetKeyInput.value += button.textContent;
            }
        });
    });

    const generateVirtualKeyboard = () => {
        const numbers = Array.from({ length: 9 }, (_, i) => i + 1);
        numbers.push(0);
        shuffleArray(numbers);
        const cleanButton = document.createElement('button');
        cleanButton.textContent = 'Limpiar';
        cleanButton.addEventListener('click', () => {
            internetKeyInput.value = '';
        });
        virtualKeyboard.innerHTML = '';
        numbers.forEach(number => {
            const button = document.createElement('button');
            button.textContent = number;
            button.addEventListener('click', () => {
                internetKeyInput.value += number;
            });
            virtualKeyboard.appendChild(button);
        });
        virtualKeyboard.appendChild(cleanButton); // Agregar el botón "Limpiar" al final
    };

    generateVirtualKeyboard();

    const generateCaptchaImage = () => {
        currentCaptcha = generateCaptcha();
        captchaContainer.innerHTML = `<img src="captcha.php?captcha=${currentCaptcha}" alt="Captcha">${currentCaptcha}`;
    };

    captchaRefreshButton.addEventListener('click', generateCaptchaImage);

    generateCaptchaImage();
});

function shuffleArray(array) {
    for (let i = array.length - 1; i > 0; i--) {
        const j = Math.floor(Math.random() * (i + 1));
        [array[i], array[j]] = [array[j], array[i]];
    }
}

    \end{lstlisting}
    \subsubsection{Ejecución de la página web}
    Procederemos a mostrar la pagina web bancaria, donde nos pedirá ingresar algunos datos y seleccionar algunas opciones, ademas el teclado virtual, algo importantes es que el teclado puede variar sus botones a distintas direcciones, por ultimo el captcha es aleatorio.

    \begin{figure}[H]
    \centering
    \includegraphics[scale=0.46]{Banca web.png}
    \caption{Ejecución de la banca web} %
    \label{fig:imagen} 
    \end{figure}

    \subsection{Pregunta}
    a) Explique una herramienta para ofuzcar código JavaScript
    UglifyJS es una herramienta de código abierto que se utiliza para minificar y ofuscar código JavaScript. La minificación es el proceso de eliminar espacios en blanco, comentarios y otros caracteres no esenciales para reducir el tamaño del archivo. La ofuscación, por otro lado, es el proceso de modificar el código para que sea más difícil de entender para los humanos, pero que siga siendo funcional para las computadoras.

    b) Muestre un ejemplo de su uso en uno de los ejercicios de la tarea.
    Procederemos a mostrar un ejemplo del código ofuscado con Banca de internet.js
    \begin{lstlisting}
        document.addEventListener("DOMContentLoaded",function(){const e=document.querySelector(".captcha"),t=document.getElementById("captchaInput"),n=document.getElementById("captchaRefreshButton"),o=document.getElementById("submitButton"),r=document.getElementById("successModal"),d=document.getElementById("reloadButton"),a=document.getElementById("cardType"),c=document.getElementById("cardNumber"),l=document.getElementById("documentType"),u=document.getElementById("documentNumber"),s=document.getElementById("internetKey"),i=document.querySelector(".virtual-keyboard"),f=function(){const e="ABCDEFGHIJKLMNOPQRSTUVWXYZabcdefghijklmnopqrstuvwxyz0123456789";let t="";for(let n=0;n
    \end{lstlisting}
    \section{Repositorios}
    \subsection{Development}
    https://github.com/kevin1go-tech/CodePlusKevin.git
    \subsection{Production}
    https://github.com/kevin1go-tech/CodePlusKevin.git
 \end{document}